
% This is file `seuthesix.tex',
% This file is the source of the documentation of the `seuthesix' class.
% Copyright (c) 2016 James Fan, email: zhimengfan1990@163.com
% License: GNU General Public License, version 3
%This file is part of ``seuthesix'' package.
%``seuthesix'' is free software: you can redistribute it and/or modify
%it under the terms of the GNU General Public License as published by
%the Free Software Foundation, either version 3 of the License, or
%(at your option) any later version.
%``seuthesix'' is distributed in the hope that it will be useful,
%but WITHOUT ANY WARRANTY; without even the implied warranty of
%MERCHANTABILITY or FITNESS FOR A PARTICULAR PURPOSE.  See the
%GNU General Public License for more details.
%
%You should have received a copy of the GNU General Public License
%along with this program.  If not, see <http://www.gnu.org/licenses/>.

\documentclass[openany,masters]{seucata}
\usepackage{hologo}
\usepackage{booktabs}
\usepackage{pdfpages}
\usepackage{fancyvrb}
\usepackage{graphicx}
\begin{document}
\categorynumber{000} 
\UDC{000}            
\secretlevel{公开}   
\studentid{000000}   
\title{欢迎神仙们引用我的模板}{求星星}{Welcome my God}{STAR}
\author{Jaylen Cheng}{jaylencheng.github.io}
\advisor{爱因斯坦}{教授}{Donald E. Knuth}{Prof.}
\coadvisor{费曼}{副教授}{Leslie Lamport}{Associate Prof.} 
\degreetype{神学}{Master of \TeX}
\major{神的产生专业}
\submajor{天使降临系}
\defenddate{\today}
\authorizedate{\today}
\committeechair{朗道}
\reviewer{Newton}{Gauss}
\department{东南大学Cata学院}{School of Cata in SEU}
\seuthesisthanks{本课题的研究获God bless Cata project 赞助:%
\url{jaylencheng.github.io}}
%\makebigcover
\makecover


\begin{abstract}{\TeX, \LaTeX, 文档类, 毕业设计论文}
本文介绍如何使用seucata 文档类撰写东南大学毕业设计论文。
\end{abstract}
\begin{englishabstract}{\TeX, \LaTeX, document class, thesis/dissertation}
This work presents an introduction of how to use seucata document class to 
typeset the Graduation design (thesis) of Southeast University.
\end{englishabstract}

\tableofcontents
\listofothers

\mainmatter


\chapter{绪论}
\verb+seucata.cls+ 提供了符合规范的东南大学本科毕业设计论文的\LaTeX 模板。
模板的格式尽量满足东南大学教务处的要求,当然由于水平有限其中错
漏在所难免,我们欢迎东大的 \LaTeX{er} 一起参加开发和完善。如果您对开发和完善seucata
感兴趣、有任何想法或建议,请与Cata联系。该项目主页GitHub:
\url{https://github.com/jaylencheng/SEU-Graduation-Design-Thsis}。
联系方式:Email(jaylencheng@outlook.com)or网页留言:
\url{https://jaylencheng.github.io}。

本模板基于早期许元同学等人发布的\verb+seuthesix.cls+,其版本见网址:
\url{https://github.com/zhimengfan1990/seuthesix}。
由于其中只有硕博士论文的模板,个人将其修改为符合本科毕业设计内容的模板。


\section{版权声明}

这一程序是自由软件,你可以遵照自由软件基金会发布的《GNU 通用公共许可证
条款第三版》来修改和重新发布这一程序,或者 (根据您的选择) 用任何更新的版本。
发布这一程序的目的是希望它有用,但没有任何担保。甚至没有适合特定目的的隐含
的担保。更详细的情况请参阅《GNU 通用公共许可证》
\footnote{\url{http://www.gnu.org/licenses/gpl.html}}

\section{版本历史}
\begin{description}
\item[1.0] 2016/01/11,基于\seuthesis 构建新的\seuthesix 文档类。放弃\seuthesis 
的xeCJK方案 ,直接采用\verb+ctexrep.cls+ 构建\verb+seuthesix.cls+。
本项目为对本科学位论文的支持。
\end{description}
                   
\chapter{下载和安装}  
\section{下载}     
本项目最新源码可以到本项目在 GitHub 中找到,访问
\url{https://github.com/jaylencheng/SEU-Graduation-Design-Thsis}下载。

\section{安装}
将宏包中的文件放在当前工作目录(与 \verb+.tex+ 文件放在同一目录下)即可,
当然也可以安装到 \TeX 系统
中,不过需要注意是参考文献样式文件 \verb+.bst+ 必须置于 \verb+TEXMF/bibtex/bst+ 目录或子目
录下。

本模板在 \TeX{Live} 2015(Windows 7 和 Linux Mint 17.1 环境下) 通过\hologo{XeLaTeX} 编译通过。注意,若在Linux 上编译,
由于Linux缺少所依赖的字体(宋体、黑体、楷体、Times New Roman),因此需要用户自己安装这些字体,否则编译时会报错。
将Windows中的字体复制到Linux中安装即可。

原则上,只要安装了\TeX{Live} 2015和宋体、黑体、楷体、Times New Roman等字体,本模板也支持OS X。(已经做过OS X 上的测试。)

如有您在使用中有任何问题,欢迎与我们联系。

\chapter{使用说明}
东南大学本科毕业设计论文应包括如下部分:\\
\begin{enumerate}
\itshape
\item 中文封面
\item 中文页面
\item 英文页面(可选)
\item 论文独创性声明和使用授权声明
\item 中文内容提要及关键词
\item 英文内容提要及关键词
\item 目录
\item \fbox{符号、变量、缩略词等本论文专用术语的注释表}(可选)
\item 正文
\item \fbox{致谢}
\item 参考文献
\item  \fbox{附录}
\end{enumerate}
并按此顺序排列,其中\fbox{加方框的条目}为可选。

\section{模板整体框架}
使用 \seuthesix 模板的整体框架如下所示,其中\verb+<...>+表示可替换的文本(replaceable text)。
{\color{magenta}
\begin{verbatim}
\documentclass[openany,masters]{seucata}
\usepackage{hologo}
\usepackage{pdfpages}
\begin{document}
\categorynumber{000} 
\UDC{000}            
\secretlevel{公开}   
\studentid{000000}   
\title{欢迎神仙们引用我的模板}{求星星}{Welcome my God}{STAR}
\author{Jaylen Cheng}{jaylencheng.github.io}
\advisor{爱因斯坦}{教授}{Donald E. Knuth}{Prof.}
\coadvisor{由乃}{副教授}{Leslie Lamport}{Associate Prof.} 
\degreetype{神学}{Master of \TeX}
\major{神的产生专业}
\submajor{天使降临系}
\defenddate{\today}
\authorizedate{\today}
\committeechair{朗道}
\reviewer{Newton}{Gauss}
\department{东南大学Cata学院}{School of Cata in SEU}
\seuthesisthanks{本课题的研究获God bless Cata project 赞助:%
\url{jaylencheng.github.io}}
%\makebigcover
\makecover
\begin{abstract}{<\TeX, \LaTeX, 文档类, 学位论文>}
<本文介绍如何使用\seuthesix 文档类撰写东南大学学位论文。>
\end{abstract}
%生成中文摘要和关键词
\begin{englishabstract}{<\TeX, \LaTeX, document class, thesis/dissertation>}
<This work presents an introduction of how to use \seuthesix document class to 
typeset the thesis/dissertation of Southeast University.>
\end{englishabstract}
%生成英文摘要和关键词
\tableofcontents%生成目录
\setnomname{<术语表名称>}%设置术语表的名称,用于\listofothers
\listofothers%生成图、表等目录,没有可以不写

\mainmatter
%开始正文

\chapter{<绪论>}
\section{<研究背景>}
\section{<本论文的工作>}
...


\chapter{<...>}
\section{<...>}
...


...


\chapter{<全文总结>}
...

\acknowledgement
<感谢每一个给予帮助的人...>
致谢,没有可以不写

\thesisbib{<bib文件名>}
参考文献

\appendix
\chapter{<...>}
...

\chapter{<...>}
...
%附录部分,没有可以不写

\resume{<作者简介>}
<简介内容...>
%作者简介,没有可以不写

\end{document}
%文档到此结束
\end{verbatim}
}

\chapter{基本语言}
\section{Introduction}

This template is based on the standard \LaTeX{} article class, which means you can pass the arguments of article class to it (\verb|a4paper|, \verb|12pt| and etc.).

\section{Font Settings}
I change the default article font computer modern to \verb|newtx| series, and the default font size is set to \verb|11pt|.

\begin{itemize}
	\item \verb|newtxtext| package for text font, similar to times new roman font.
	\item \verb|newtxmath| package for math font, close to \verb|times| and \verb|mtpro2| packages.
	\item \verb|newtxtt| package for typewriter font, with option \verb|scale = 0.8|.
\end{itemize}

These packages operate perfectly but are inappropriate for big operators, for example \verb|\sum| and \verb|\prod|, thus, I change these operators back to computer modern font. Equation~\eqref{eq:binom} shows the effects of these fonts:
\begin{equation}
(a+b)^{n} = \sum_{k=0}^{n} C_{n}^{k} a^{n-k} b^k \label{eq:binom}
\end{equation}



The \verb|\linespread| (controls line spacing) is set to 1.3, and I use \verb|microtype| to improve the font justification. \verb|type1cm| package is used to remove the font shape and font size warning messages.

\section{Custom Commands}

I don't change any default command or environment, which means you can use all the basic \LaTeX{} commands and environments as before.  Besides, I define 3 commands
\begin{enumerate}
	\item \verb|\email{#1}|: create the hyperlink to email address.
	\item \verb|\figref{#1}|: same usage as \verb|\ref{#1}|, but start with label text <\textbf{Figure n}>.
	\item \verb|\tabref{#1}|: same usage as \verb|\ref{#1}|, but start with label text <\textbf{Table n}>.
\end{enumerate}{}

\section{Table}
I strongly recommend you to use the \verb|booktabs| package in your paper. It adds three commands to make the table prettier, ie. \verb|\toprule|, \verb|\midrule| and \verb|\bottomrule|. Here is an example.

\begin{table}[!htbp]
	\small
	\centering
	\caption{Regression Result Example}
	\begin{tabular}{lll}
		\toprule
		& \multicolumn{1}{c}{(1)} & \multicolumn{1}{c}{(2)} \\
		& \multicolumn{1}{c}{price} & \multicolumn{1}{c}{price} \\
		\midrule
		mpg   & \multicolumn{1}{c}{-238.9***} & \multicolumn{1}{c}{-49.51} \\
		& \multicolumn{1}{c}{(53.08)} & \multicolumn{1}{c}{(86.16)} \\
		weight & \multicolumn{1}{c}{} & \multicolumn{1}{c}{1.747***} \\
		& \multicolumn{1}{c}{} & \multicolumn{1}{c}{(0.641)} \\
		constant & \multicolumn{1}{c}{11,253***} & \multicolumn{1}{c}{1,946} \\
		& \multicolumn{1}{c}{(1,171)} & \multicolumn{1}{c}{(3,597)} \\
		observations & \multicolumn{1}{c}{74} & \multicolumn{1}{c}{74} \\
		R-squared & \multicolumn{1}{c}{0.220} & \multicolumn{1}{c}{0.293} \\
		\midrule
		\multicolumn{3}{l}{\scriptsize Standard errors in parentheses} \\
		\multicolumn{3}{l}{\scriptsize *** p<0.01, ** p<0.05, * p<0.1} \\
	\end{tabular}%
	\label{tab:reg}%
\end{table}%



\section{Graphics}
To include a graphic, you can use figure environment as usual. You can put all your images in the sub directories (\verb|./image/|, \verb|./img/|, \verb|./figure/|, \verb|./fig/|) of your current working directory.

\begin{Verbatim}[tabsize=4,frame=single,baselinestretch=1]
\begin{figure}[!ht]
\centering
\includegraphics[width=0.6\textwidth]{./figures/seu-color-logo.png}
\caption{The Relationship between MPG and Weight\label{fig:mpg}}
\end{figure}
\end{Verbatim}
\begin{figure}[!ht]
	\centering
	\includegraphics[width=0.6\textwidth]{./figures/seu-color-logo.png}
	\caption{The Relationship between MPG and Weight\label{fig:mpg}}
\end{figure}

\subsection{Bibliography}
This template uses Bib\TeX{} to generate the bibliography, the default bibliography style is \verb|aer|. ~\cite{Chen2018} use data from a major peer-to-peer lending marketplace in China to study whether female and male investors evaluate loan performance differently. You can add bib items (from Google Scholar, Mendeley, EndNote, and etc.) to \verb|wp_ref.bib| file, and cite the bibkey in the \verb|tex| file.











\chapter{注意事项}
\section{文献引用}
根据要求,文献引用应为数字标签(numerical label),上标形式。但是有时也会用到正常形式,即非
上标形式,用于正文叙述。为此分别提供了两个命令来实现。

{\color{magenta}%
\begin{verbatim}
\cite{<citation_key>}
\end{verbatim}
}

用于实现上标的数字形式文献引用\cite{knuth}。

{\color{magenta}%
\begin{verbatim}
\citen{<citation_key>}
\end{verbatim}
}

用于实现正常(非上标,normal)的数字形式文献引用\citen{mittlebach}。

\section{参考文献格式}
见\verb+seuthesix.bst+的文档: 本文第\ref{bst}章。

\section{图表格式处理}
图名、表名字体字号已经有文档类 设定好,用户无需再次设定。但是,用户需要让它居中。图名位于图下方,
表名位于表上方。图片文件可直接置于当前工作目录,也可置于当前工作目录的\texttt{figures}子目录下(用户根据需要,自己创建该子目录)。
图片名称只需要给出名称和扩展名,无需给出完整的路径。图\ref{logo}给出了一个图的例子。表\ref{entrytable}给出了一个表的例子。

{\color{magenta}%
\begin{verbatim}
\begin{figure}
\centering
\includegraphics[...]{...}
\caption{...}
\label{...}
\end{figue}
...
\begin{table}
\centering
\caption{...}
\label{...}
\begin{tabular}{...}
...
\end{tabular}.
\end{table}

\end{verbatim}
}

\begin{figure}
\centering
\caption{\seuthesix logo\label{logo}}
\chuhao \seuthesix
\end{figure}

\section{算法格式处理}
\seuthesix 文档类采用了\texttt{algorithm, algorithmic }两个宏包来设置算法排版格式。
详细使用方法参见这两个宏包的手册。这里给出一个简单的例子。

{\color{magenta}%
\begin{verbatim}
\begin{algorithm}
\caption{\label{algoinsight}如何使用\seuthesix 文档类}
\begin{algorithmic}[1]
\STATE if (具备一定的\LaTeX 使用经验) else (stop here)
\STATE 有耐心阅读文档
\STATE 仔细阅读本文档
\STATE 在使用中熟悉它
\end{algorithmic}
\end{algorithm}
\end{verbatim}
}

以上代码给出了算法\ref{algoinsight}的排版结果。
\begin{algorithm}
\caption{\label{algoinsight}如何使用\seuthesix 文档类}
\begin{algorithmic}[1]
\STATE if (具备一定的\LaTeX 使用经验) else (stop here)
\STATE 有耐心阅读文档
\STATE 仔细阅读本文档
\STATE 在使用中熟悉它
\end{algorithmic}
\end{algorithm}

\section{术语生成}

{\color{magenta}%
\begin{verbatim}
\nomenclature{LTE}{Long Term Evolution}
\nomenclature[noprefix]{$\mathcal{CN}(0, C)$}{协方差矩阵为$C$的循环对称复高斯分布}
\end{verbatim}
}

\nomenclature{LTE}{Long Term Evolution}
\nomenclature[noprefix]{$\mathcal{CN}(0, C)$}{协方差矩阵为$C$的循环对称复高斯分布}
术语生成借助\texttt{nomencl}宏包。以上代码例子的排版结果在术语表中。对于数学符号,为使得它
排在普通术语的后面,需要加上\texttt{[noprefix]}选项。
编译时,第一次

{\color{magenta}%
\verb+xelatex <filename>+ 
}

之后执行 

{\color{magenta}%
\verb+makeindex <filename>.nlo -s nomencl.ist -o <filename> .nls+
}

然后再次执行

{\color{magenta}%
\verb+xelatex <filename>+
}

一次。实际上,考虑到参考文献的生成,最后应该执行

{\color{magenta}%
\verb+xelatex <filename>+
}

至少两次。

\chapter{\texttt{seuthesix.bst}参考文献格式\label{bst}}
\section{简介}
\texttt{seuthesix.bst} 是符合东南大学硕士和博士毕业论文参考文献格式要求的 bibliography style。
由于该文献格式是数字标签(numeric label)格式,且无需排序。因此,它是在标准的BibTeX
格式\texttt{unsrt.bst} 的基础上发展而来。同时,\verb=seuthesix.bst=文献格式不支持\texttt{crossref}字段。
该文献格式支持中英文两种语言,采用中英文分离,分别进行处理的思想。对于中文文献的entry,
要求在\verb=.bib=文献数据库中该entry 提供\texttt{language} field
,其取值任意,但为了保持一致性,建议设为\verb+language={zh}+,或\verb+language="zh"+。

\section{支持的Entry type}
\texttt{seuthesix.bst}支持以下几种不同的entry type。
\begin{description}
\item[book] 书籍
\item[article] 期刊文章
\item[inproceedings/conference] 会议文章
\item[mastersthesis] 硕士学位论文
\item[phdthesis] 博士学位论文
\item[patent] 专利
\item[standard] 标准
\item[news] 报纸新闻
\item[misc] 杂项,包括电子媒体及其他未能识别的entry type
\end{description}

\section{各 entry type 所支持的field}
不同的entry type所支持的field 不尽相同。对于每个entry type 而言,其中的field 可分为三类:
required, optional, ignored。required field 应该必须出现,否则BibTeX 会发出warning(由bst中的output.check函数发出),
且最终得到的格式不能保证美观。
optional field 可选,若未出现,不会导致BibTeX warning。ignored field 
不起任何作用,被BibTeX 忽略。在entry type 对应的bibtex 函数(如 \verb=FUNCTION {article}=)中未使用的 field 都属于 ignored field。

下面列出各entry type所支持的field(即required field 和optional field),
其中required field使用黑体。如表\ref{entrytable}所示。

\begin{table}
\centering
\caption{\label{entrytable}不同entry type支持的field}
\begin{tabular}{|c||p{10cm}|}
\hline
\bfseries Entry type & \raggedright\bfseries Fields\tabularnewline
\hline\hline
article & \raggedright{\bfseries author, title, journal, year, } volume, number, pages, note, {{\bfseries lang}uage}\tabularnewline
\hline
book & \raggedright{\bfseries author/editor, title, }edition, translator, address, publisher, year, pages, note, {{\bfseries lang}uage}\tabularnewline
\hline
inproceedings(conference) & \raggedright {\bfseries author, title, }editor, booktitle, address, publisher,
year, pages, note, {{\bfseries lang}uage}\tabularnewline
\hline
mastersthesis  &\raggedright{\bfseries author, title, address, school, year, }note, {{\bfseries lang}uage}\tabularnewline
\hline 
phdthesis  &\raggedright{\bfseries author, title, address, school, year, }note, {{\bfseries lang}uage}\tabularnewline
\hline 
patent & \raggedright{\bfseries applicant\footnotemark, title, }littype\footnotemark,
country, pid\footnotemark, year, month, day, 
note, {{\bfseries lang}uage}\tabularnewline
\hline
standard & \raggedright{\bfseries author, title, }stdcode\footnotemark, address, publisher, year, 
note, {{\bfseries lang}uage}\tabularnewline
\hline
news & \raggedright{\bfseries author, title, newspaper, }year, month, day, pages, note, {{\bfseries lang}uage}\tabularnewline
\hline
misc &\raggedright{\bfseries }author, title, url, year, month, day, note, {{\bfseries lang}uage}\tabularnewline
\hline
\end{tabular}
\end{table}
\footnotetext[1]{专利申请人}
\footnotetext[2]{专利文献类型}
\footnotetext[3]{专利号}
\footnotetext[4]{标准号}
需要注意的是,若为英文参考文献,则language field 可以省略,而对于中文参考文献language field 则必须出现在\verb=.bib= 文献数据库中。
其取值任意,但建议设为\verb+language={zh}+,或\verb+language="zh"+,以保持一致性。
这就是表\ref{entrytable}中的{{\bfseries lang}uage}一半黑体一半非黑体的原因。
参考文献部分给出了很多文献格式的例子。

\nocite{komine2004fundamental}
\nocite{dimitrov2015principles}
\nocite{fujimoto2014fastest}
\nocite{ieee2012ieee}
\nocite{irdawebsite}
\nocite{vlcnews}
\nocite{pt}
\nocite{thesis:a}
\nocite{thesis:b}

\chapter{全文总结}
本文主要介绍了如何使用seucata \LaTeX 文档类来对东南大学硕士与博士学位论文进行排版。

\acknowledgement
感谢每一位支持seucata 的人!

\thesisbib{seuthesix}

\appendix

\end{document}
